\section{$\omega$ and $\Omega$ Types}

\label{sec:omega_types}

The $\omega$ type and their generalization \tomega types, serve to model unrestricted multiplicty. 
An unrestricted multiplicity as one that can create any number of linearly bound terms, which is written by Girard as \code{!} (``of course''), $\omega$ in Idris, and $t \code{[]}$ in \granule \cite{linear_logic,idris,granule}. 
Further, in other works of embedding a linear logic system with unrestricted multiplicity into a more restrictive system, the system of natural number was extended to conatural numbers, which contain $\infty$.

Here, we propose \tomega types, and a more specific form of them, $\omega$ types, which are written in \thislib as \code{omega}.
We will first look at $\omega$ types, as they are a fair bit simpler than $\Omega$ types.

\begin{definition}
	$\omega$ is a type alias which takes a type and a witness of that type as arguments, we define it as $\omega \ t \ w \deq \fn1 n : \nat . \tm n t w$
\end{definition}

We can thence, if we have $x : \omega t w$, get any number of bindings of the value $w$ of $t$.
This can then be used to model multipliciites.
Firstly, let us note how we might construct these.

\begin{definition}
  \ident gen goes from a value in a linear unrestricted multiplicity, $!_*$, and a value of $\omega$.
  Spefically, \ident gen has the type signature $\code{gen} : \fn1 x : (!_* t) . \omega t (\code{unrestricted} x)$..
  It has the definition of \missingcode .
\end{definition}

This also us to unpack a value of unrestricted multiplicity into some number of linear bindings.
\todo{Using omega types}

\subsection{$\Omega$ types}

Whilst $\omega$ types allow us to model general unrestricted multiplicities, they dont allow for for restricted multiplicity polymorphic functions.
For instance, consdier the function in \granule \ident mapMaybe, \marker type .
We define this with $\omega$ and \tmu types as \marker def where we define the erased value \marker MapMaybeW .

However, this would be overly specific. 
Considering the \granule equivalent, it is clear what the problem is. 
Namely, we will use \ident f at most once, so we don't need \emph{any} number of usages of \ident f .
We need at most one instance of \ident f .
This has real world consequences.
For instance, if we have $\code{x} : \nat$, we might construct $f : \fn1 \_ : \nat . \nat$, $f y \deq x + y$.

In addition, we can discard this by merely doing $f \code{Zero}$, which then has the type $\nat$, which we can freely discard.
So, we have a way to get either 0 or 1 instances of $f$, but not, in general, to get any number of these.

This isn't a mu type, as there are two choices, 0 and 1, for the multiplicity.
However, it also isn't an $\omega$ type, as we need not be able to produce any number of values.
To solve this dilemena, we introduce $\tomega$ types, a generalization of $\omega$ types to require only a subset of $\nat$ bindings.

Firstly, we note that is Idris, \verb|Type| and \verb|Prop| are synonymus.
We also note that then a prediciate on a type $\pred{a}$, is just equivalent to $\fn* \_ : a . \tpe$ \cite{proofs_and_types}.
We can likewise define a number of relevant operations on $\pred{a}$, which are given in more detail in \thislib \verb|Data.Mu.Maps|.

One of the more interesting facts about predicates is that they are contravariant functors.
% We define the \verb|contramap| function with the name \code{PMap}, and with signature $\code{PMap} : \fn* \_ : (\fn* \_ : a . b) . \fn* \_ : \pred{b} . \pred{a}$.
We also define in a slightly less interesting fashioin a number of other general predicate functions.

We in particular focus our attetetion on \pred{\nat}, which we can use to define \tomega as follows:

\begin{definition}
  \tomega is a polymorphic type over a predicate on \nat, $p$, a given type $t$, and a witness of that type $w$, $\tomega : \fn* p : \pred{\nat} . \fn* t : \tpe . \fn* w : t . \tpe$.
  We define $\tomega$ as $\tw p t w \deq \fn1 n : \nat . \afa0 \code{prf} : (p n) . \tm n t w$.
\end{definition}

So, to get a certain number of bindings of $w$, we must provide a proof that the given number of bindings satisfies some arbitrary predicate $p$.
\todo{Finish, among things, work on 3A}
