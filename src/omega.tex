
\section{$\Omega$ Types}

\label{sec:omega_types}

The principle use of \code{Form} here is as part of the $\Omega$ construction, which uses it to constraint over mutliplicity, thereby giving us an equivalent of \granule effect formulas,

\subsection{Poly-multiplicative Judgments}

\begin{definition}
	\label{def:Omega_ty}
	$\Omega$ is a type indexed by a formula, $\phi$, a type $t$, and a witness of that type, such that $\tw \phi t w \deq_0 \tw_n \phi t w$
\end{definition}

That is, an $\Omega$ designates a function from a natural number to a certain number of $t$.
However, this is often unedsierable, so we instead efine a helper function, reify, of the type $\fn_1{\tw \phi t w}{\fa_1[n]{\nat}{\afa_0{(n \in \phi)}{\tm n t w}}}$, which, allows for us to instead consider if the number in is in the ``solution set'' as opposed to a value projected to.
  
The most simple form of $\Omega$ is that where $\phi$ is \code{FVar} and thus is simply, $\tw_n {\code{FVar}} t w$, or simply $\tw* n t w$, which we specifically call $\omega t w$, and $\omega t w \deq_0 \tw* n t w$ 

%END OF SET E