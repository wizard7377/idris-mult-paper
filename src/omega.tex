\section{Omega Types}

While $\musym$ types allowed us to construct a way to model non-linear restricted multiplicites, they don't give us the full power of a system like Granule.
This is because a system like GrTT admits arbitrary multiplicity polymorphism.

The simple solution to this is to bundle multiplicity polymorphism, using effect formulas, into a single type.
We call this type $\Omega$, and it generalizes mu to arbitrary effect formulas.

\subsection{Free and Forgetful Bindings}
\label{sec:omega_def}


There are two ways that we can define an $\Omega$ type, depending on whether we want it to be able to rember its structure or not.
These both have the exact same type signature, both having the types of $\fn<1>[p]{\vform} \fn<1>[t]{\tpe} \fn<1>[w]{t} \tpe$.
We call the forgetful version, which is a type family, $\Omega '$ (\code{Omega'} in the library) and the free version, which is a datatype, $\Omega$ (\code{Omega} in the library).
Both of these are intended, for the arguement listed above, a way to represent $\tm{n}{t}{w}$, where $n$ is any solution to $p$.

\subsubsection{Forgetful Omega}

The first of the is the forgetful way.
This way is the simpler of the two.
It simply is defined as follows:

\begin{figure}
  \begin{align}
    \label{eq:omega'_def}
    $\Omega' \ p \ t \ w \deq \fa<1>[n]{\nat} \afa<0>{
    \left(
    \fsolve{p}{n}
    \right)} \tm{n}{t}{w}$
  \end{align}
\end{figure}

This is perhaps simplest for the special case of $p$ being $()$.
When $p$ is a free variable, the second arguement trivially exists, and therefore this just becomes $\fa<1>[n]{\nat} \tm{n}{t}{w}$.
That is it becomes a way to generate however many bindings we would like.
Then, by varying $p$, we restrict the set of possible bindings to only those whose number is a solution to $p$, so for instance, $\Omega [0,1] t w$ models the notion of a value that may or may not be used.

\subsubsection{Free Omega}



%%% Local Variables:
%%% mode: LaTeX
%%% TeX-master: "../main"
%%% End:
