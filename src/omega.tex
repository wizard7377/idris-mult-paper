
\section{Omega Types}

While mu types allowed us to generalize linear quanties to graded ones, we have a construction, $\Omega$, that generalizes mu types to arbitary formulas.
We do this by specifing a quantity formula for the type, rather than a specific quantity.

\begin{definition}[Omega definition]
  $\Omega$ is a type indexed on a \code{Form}, a type, and a witness of that type such that
  $\tw{p}{t}{w} \deq \fn<1>[n]{\nat} \afa<0>{n \in p}{\tm n t w}$
\end{definition}

That is, it maps a number that satisfies the property of being a solution to a formula to that many bindings of that value.

\subsection{Basic Properties}
\label{sec:omega_prop_basic}

One of the most signaficant properties of $\Omega$ is that it can be ``weakened''.
That is, if a given formula $p$ is ``in'' $q$, then there exists a function from $\tw{q}{t}{w}$ to $\tw{p}{t}{w}$, which we call \code{weaken}.
The reason that this must exist is apparent upon expanding the types, whence we get $\code{weaken} : \afa<0>{p \subseteq q} \fn<1>{(\tw*[m]{q}{t}{w})}{(\tw*[n]{p}{t}{w})}$, which is trivial
\todo{Motavating example}

\subsection{Operations on Omega}
\label{sec:omega_ops}

\subsection{Completeness and Uniqueness of Omega}
\label{sec:omega_complete_unique}

There are two very important facts about omega types, specifically with reference to mu types.
Namely, they both revolve around the idea that omega types can serve as an exact model for mu types.
The first of these is the fact that forall $\tm n t w$, there is a linearly isomorphic $\Omega$

\subsection{Infinite Lazy Copies}
\label{sec:omega_lazy}

%%% Local Variables:
%%% mode: LaTeX
%%% TeX-master: "../main"
%%% End:
