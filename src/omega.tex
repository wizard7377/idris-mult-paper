\section{$\omega$ and $\tomega$ Types}

While all this is useful, we still need to be able to incorperate the idea of "unrestricted" bindings into our set of constructs.
This is the goal of the $\omega$ and $\Omega$ constructions, to provide model for almost all multiplicities, without extending the core language.

\subsection{$\omega$}

The $\omega$ construction is quite simple.
It takes advantage of the fact that there isn't a distinction between a value and a way to generate it.
So, we don't need an actual variable number of bindings, we merely need a way to generate for any $n$ some $n$ bindings. 
This is all $\omega$ is, a continuation on the number of bindings.

So, $\omega t w \equiv ((n : \nat) \to M n t w)$ \todo{add more info on this}
\subsection{$\omega$ as $!_*$}

\subsection{$\Omega$ as a generalization of $\omega$}

\subsection{$\tomega$ as a generalization of $M$}

\subsection{$M$ to $\tomega$}