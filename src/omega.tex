\section{$\omega$ and $\tomega$ Types}

One of the more notable constructions in Girard's linear logic is the unrestricted, "of course", construction \cite{linear_logic}.
This allows one to abstract infinitly many values of a given statement.
In Granule, this is written as $[0 \dots \infty]$.

We define a way to model statements these unrestricted bindings using $M$ bindings, which we call \tomega, or \ident Omega , which is defined in \ref{lst:Omega_def}. 

\begin{listing}
	\begin{minted}{idris}
		0 Omega : (p : (Nat -@ Nat)) -> (t : Type) -> (w : t) -> Type
		Omega p t w = (1 x : Nat) -> (Mu (p x) t w)
	\end{minted}
	\caption{Definition of the Omega type}
	\label{lst:Omega_def}
\end{listing}

However, for now, we will restrict our view to $\omega$ types, which are defined as $\omega t w \deq \Omega (\lambda x . x) t w$.
When we expand this, we get $\omega t w \deq \eomega{(\lambda x . x)}{t}{w}$, or, upon beta reduction, $\omega t w \deq \fa1 n : \nat . {\tm n t w}$
Namely, this allows us to create any number of bindings of $w$.

In this sense, we have an unrestricted value of $w$, as, whenever we need a value of a concrete $M$, we can just evaluate the continuation for some $n$.
This is hinted at by the Granule syntax, for any $n$, we can create exactly $n$ bindings.
We've already created a construction on $\omega$, namely, \ident genMu .
Before, we listed $\ident genMu : \fn1 x : {!_* t} . \fa1 n : \nat . {\tm n t x.{\ident unrestricted }}$, however, we can replace the $\fa1 n : \nat . {\tm n t x.{\ident unrestricted }}$ with $\omega t w$, thereby yielding $\ident gen : \fn1 x : {!_* t} . \omega t {x.{\ident unrestricted }}$.

\todo{Splitting Omega}
\subsection{General $\Omega$}

\subsection{Countable Sets}

% https://proofwiki.org/wiki/Cartesian_Product_of_Countable_Sets_is_Countable

\subsection{Resource Algebras}