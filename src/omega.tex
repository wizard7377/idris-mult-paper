
\section{Omega Types}

Omega types allow us to generalize mu types to bindings that have mutiple possible values.
For instances, in the \granule{} binding \verb|x : t [2*c]|, the binding has a variable multiplicity given by the effect formula $2 * c$ \cite{dep_mult_dep_lin}.
This allows for Granule to have, say, a function \code{mapMaybe} which has the form $\fn<0..1>{(\fn<1>{a} b)} {\fn<1>{(\code{Maybe} a)} (\code{Maybe} b)}$.

Of course, this is just one example of many of the potential utility of such a system, perhaps the most interesting of which is modeling the idea of optional ownership.
We propose $\Omega{}$, which model such bindings of variable multiplicity using a continuations on the exact number of bindings 

$\Omega$ types allow for bindings that have multiplicity polymorphism.
The simplest example of this is a binding that may or may not be used.
In \granule{} such bindings are created by allowing for effect formulas to serve as multiplicities 

\subsection{Extended Mu}
\label{sec:omega_def}

\begin{definition}[Omega types]
  $\Omega$ is an erased function that takes a \code{Form} as an arguement, as well as a type, $t$, and a witness of $t$, which, altogether, has the singature $\Omega : \fn{\code{Form}} \fn[t]{\tpe} \fn{w} \tpe$.
  Its definition is $\tw{\phi}{t}{w} \deq \tw*[n]{\phi}{t}{w}$
\end{definition}

This is simplest understood by example.

The easiest form of this is $\tw{\code{FVar}}{t}{w}$, which expands to the type $\tw*[n]{\code{FVar}}{t}{w}$.
Per \ref{lem:gen_var}, this becomes simply $\fn<1>[n]{\nat} \tm n t w$.
Thereby, this is simply a mapping from any number of bindings to that many bindings of the form $w : t$.

Another simple form of $\Omega$ is that where the formula is some $\code{FVal}$.
This type, given that the specific number is $m$, expands to $\tw*[n]{\apply{FVal}{m}}{t}{w}$.
Because $\fsolve{\code{FVar m}}{n}$ only exists if $n \peq m$, we know that this will simply be equivalent to $\tm m t w$.

\subsection{Operations on Omega}
\label{sec:omega_operate}

Given the fact that $\Omega$ attempts to generalize $M$, it stands to reason that each of the operations on \musym{} have equivalents on $\Omega$.

This, unfourtanetly, is only partially true.
This is because while operations that serve to ``combine'' \omegasym{} with itself generally exist, operations that ``split'' \omegasym{} into many \omegasym{} do not exist.
This is because of the fact that each \omegasym{} is quite crucially a linear function, we can only use it once.
However, let us assume we had a $\fn<1>{\tw{(\phi + \psi)}{t}{w} (\tw{\phi}{t}{w} \times^1 \tw{\psi}{t}{w})$

\subsection{Infinite Lazy Copies}

\label{sec:omega_lazy}

%%% Local Variables:
%%% mode: LaTeX
%%% TeX-master: "../main"
%%% End:
