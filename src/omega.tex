\section{$\omega$ Types for Simplicity and Performance}

Before, it was stated that one of the main points of $M$ types was to create model for certain unrestricted multiplicities using only linear ones.
This is done through the usage of $\omega$ types, which allow for an arbitrary number of copies to be created:

\begin{definition}
	$\omega$ is a function from types to types.
	It is defined as follows:
	$\omega \ t \equiv \{n : \nat\} \proc t^n$
\end{definition}

That is, it is a continuation that, when given (either implicitly or explicitly) a value for $n$, produce value of $t$ with multiplicity $n$.

\begin{note}
	Even though they aren't of the same form, we also use $t^\omega$ and notation for $\omega t$
\end{note}

\subsection{Using $\omega$}

The simplist way to use $\omega$ is to apply it. 
So, if $x : t^\omega$, then $x \{1\}$ has type $M^1$.

Like other continuations, however, the real power of $\omega$ is in composing it.
If we have a function $f : a^n \proc b^n$, and a $x :_1 a^\omega$, which expands to $x : \{n : \nat\} \proc a^n$, we can then perform a simple linear composition, yielding $(f \circ x) : \{n : \nat \} \proc b^n$, which can then be reduced to $(f \circ x) : b^\omega$.

\subsection{Generating $\omega$}

Given the fact that both $!_*$ and $\omega$ both describe unrestricted multiplicites, it might seem natural that there is a linear isomorphism between them.
Indeed, there is a way to get $\omega t$ from $!_* t$, and it bears the name \verb|gen|.
Its implementation may be found in \ref{genW}

\begin{listing}
	\label{genW}
	\begin{minted}{idris}
		genW : a -> omega a 
		genW x = case n of
			Z => MZ
			(S k) => MS x (Delay (gen x {n = k}))
	\end{minted}
	\caption{$gen$ for $\omega$}
\end{listing}

However, the opposite function does not exist. 
This is because it would require the ability to "combine" an "infinite" number of linear values to get one unrestricted one. 
A type of such a procedure in Idris might be $(a \proc (a *_1 b)) \proc (a \proc (a *_1 (!_* b)))$. 
That is, given a way to recursivly generate values of $b$ from $a$, we can get $b$ unrestricted from $a$.

However, this does not exist, and also requires a proof that all the values generated by $a$ are indistinct, so, in general, we can only ever have a map of unrestricted to $\omega$.

\subsection{Generalized $\Omega$}

In QTT, it there exists a mapping from $(a \proc b) \proc (a \to b)$. 
That is, there is a functor, $F$, that maps the category of types with linear morphism to types with \emph{any} morphism.

Essientally, if we have a linear function, we can reason about it as a regular function.
It then raises the question: given that $M$ serves as a base for "generalizing" finite multiplicites, and $\omega$ serves as one for unrestricted ones. 
It then seems reasonable to ask, as $M$ form a category, do $\omega$? 
Is there a functor from $M$ to $\omega$?

There is, but it requires a small extension to our system.
To see why, let us consider trying to apply the constructor $\nsum$ to $\omega$. 
$\nsum : t \proc M_n t \proc M_{n + 1} t$, partially apply it, and we get $(x \nsum) : M_n t \proc M_{n+1} t$.
When we try using the continuation of $\omega t$ to this, we end up getting the type ${n : \nat} \proc M_{n+1}$. 
This \emph{isn't} a $\omega$, we can't create a $M_0$ from it ($0$ is the successor to no number).

The solution to this is $\Omega$ types (written \verb|W| in Idris). 
The $\Omega$ takes an additional implicit arguement in its type.
Its form is: $\Omega \ \{p\} \ t \equiv (n : \nat) \proc M_{p n} t$, where $p$ is a implicit arguement of the type $\nat \proc \nat$. 
What $\Omega$ does is it "limits" what multiplicities it can be by defining a projection on natural numbers. 
Note that this must be total, that is, we still have to have $p$ be complete over $n$. 
We can, however, for instance in the previous case, compose $t^\omega$ and $t^n \proc t^{n + 1}$ to get $\{n : \nat\} \proc t^{n + 1}$, which we can reduce to $\Omega_{\lambda x . (x + 1)} t$.

It should also be noted that $\omega$ is a specific form of $\Omega$, specifically, $\Omega_{\lambda x . x}$, which just is over the set of $\nat$ unchanged

