\section{Exponential Types}

Linear exponential types have been noted previously to be equivalent to a certain number of bindings of a value.
Indeed, this is why they are called \emph{exponential} types, as, say, $\code{String}^3$ models $\code{String} \times \code{String} \times \code{String}$.
Here, we construct $\code{Exp}$, which models exponential types using $\Omega$ types abstracted over their witness.
This allows us to transform the $\Omega$, which requires a witness of the value, to something more closely resembling the type of graded value \emph{themselves}.

\begin{definition}[Exponential Types]
  \label{def:exp_type_core}
  \code{Exp} is a type function indexed on a formula, $\phi$, and type, $t$, with the definition of $\code{Exp} \phi t \deq \et<1>[w]{t} {(\tw{\phi}{t}{w})}$.
\end{definition}

We also, as the name suggests, employ a terse notation using the syntax sugar described for formulas earlier \ref{sec:formulaic} to write these more expressivly.

\begin{notation}
  We write $t \carat \phi$, where $\phi$ is a formula-like object and $t$ is a type for $\code{Exp} \phi t$, and, for an even simpler syntax, $t^\phi$. 
\end{notation}

This syntax allows us to write, say $t^2$ for $\code{Exp} {(\code{Given} 0 \code{FVal'} (S (S 0)))} t$, which is obviously much clearer \footnote{Obviously, this example is a little bit ridiculos, as we already have, say, $2 \peq (S (S 0))$, but it still is true that this is a more terse syntax}.

\subsection{Exponential Types as Graded Values}
\label{sec:exp_grade}

Exponential types serve the primary purpose of modeling linear values, rather than the bindings of those values.
So, for instance, the granule type $\code{String} [2]$ is modeled by $\code{String}^2$
One of the most important facts about exponential types is that they are functorial over \emph{both} their first and second arguements\footnote{With respect to $\subseteq$}.

\begin{lemma}
  Exponential types have a function ${\code{map}_{\code{Exp}}}_{2} : \fa<0>[p]{\fn<1>{t} u} \fn<1>{(\fa<0>[w_t]{t} \fn<1>{\tw{\phi}{t}{w_t}} \tw{\psi}{u}{(p w_t)})} {\fn<1>{\txp{\phi}{t}} \txp{\psi}{u}}$ \footnote{Note that when we want to disambitguate among \musym and \omegasym and $\code{Exp}$ types, we use a subscript to differentiate them}
\end{lemma}
\begin{proof}
  For ${\code{map}_{\code{Exp}}}_{2} f x$, we have $Given (p x_{1}) (\map_{\Omega} f x_{2})$
\end{proof}

Next, and slightly more interesting, is that over the second 
\begin{lemma} 
  Exponential types have a function ${\code{map}_{\code{Exp}}}_{1} : \afa<0>{\phi \subseteq \psi}{\fn<1>{\txp{\phi}{t}} \txp{\psi}{t}}$ (we call this \code{weaken})
\end{lemma}
\begin{proof}
  We simply have \todo{Finish}
\end{proof}

And therefore, we have \todo{Add bimap}.
We can also state that the initial object in the category of exponential types is $\bot^n$, as the inital category of types is $\bot$ and the inital formula is just a variable.

This shows something very important, exponential types allow us to model graded values using only linear types, but rather than modeling them as judgements, viewing them as types.
In essence, the only reason that we need the witness is to ensure that all the values are the same.
However, they maintain all the utility of terms while having the full flexibility of $\Omega$ types.

Granted, they have one major problem.
Namely, operations that split them have the same restrictions as described in \ref{fig:bad_expand}.
In addition, the combine rule has an additional restriction: the exponenets must have the same witness.
This is because we can no longer guarntee that two exponentials having the same type means that they provide the same value.

\subsection{The type of Strings Squared}
\label{sec:num_exp}

As has been noted before, linear exponential types have a very nice interpretation as numerical exponentials, namely, the following is always true

\begin{figure}
  \label{fig:num_exp}
\begin{align}
  t^{n + 1} &\liso (\sg<1>{t} t^n) \\
  t^0 &\liso \mathbf{1} \text{When $t$ is not $\mathbf{0}$} \\
  \mathbf{0}^n &\liso \mathbf{0} \\
  t^1 &\liso t \\
  \mathbf{1}^n &\liso \mathbf{1}
\end{align}

\end{figure}
Given that $\mathbf{0} = \bot$, and $\mathbf{1} = \top$.

\begin{remark}
  All the properties in \ref{fig:num_exp} are true.
\end{remark}
\begin{proof}
  The first of these is the only one that is truelly ``diffucult''. However, since we \emph{can} define splitting of finte values, this is not incredibly diffucult.
  The rest of the proofs somewhat simply follow
  \begin{itemize}
  \item $\txp{0}{t}$ simply reduces to $\txp[w][n]{0}{t}$, which 
  \end{itemize}
\end{proof}
\subsection{Inverting Types}
\label{sec:exp_inv}


%%% Local Variables:
%%% mode: LaTeX
%%% TeX-master: "../main"
%%% End:
