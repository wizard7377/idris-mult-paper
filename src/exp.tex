\section{Exponential and Existential Types}

While the $M$ and $\Omega$ types bear much insert theoretically, they, by themselves, have little practical use.
This is because as demonstrated in \ref{sec:grade_mu}, these types model the \emph{judgments} about graded modalities, not the graded modalities themselves.

Fortunately, however, we can define a simple abstraction over them that allows them to behave more like true graded modalities at the term level.
To do this, we must first introduce a linear existential type.
A regular existential type is a dependent pair type that "doesn't care" about its first argument. 
In Idris 2, we can make the that fact part of the programming by giving the first argument a multiplicity of zero.

Idris 2 actually defines an existential type, however, it has the second argument have a multiplicity of $\omega$, while we want it to have a multiplicity one.
Fortunately, the modification of \ident Exists to our type, \ident LExists , is quite trivial\footnote{Idris, however, seems to have trouble correctly linearlizing constructor accessors, so we define the actual \ident fst and \ident snd accessors separately}, and we define it in \ref{lst:def_lexists}

\begin{listing}
	\begin{minted}{idris}
		record LExists {ty : Type} (f : (ty -> Type)) where
			constructor LEvidence
			0 fst' : ty 
			1 snd' : f fst'
	\end{minted}
	\caption{The definition of \ident LExists }
	\label{lst:def_lexists}
\end{listing}

We can also define mapping like how Idris defines the mapping, with the signature described in \ref{lst:def_lex_map} \cite{idris_base}.

\begin{listing}
	\begin{minted}{idris}
		map : 
			{0 p : a -> Type} -> 
			{0 q : b -> Type} -> 
			{0 m : (a -> b)} -> 
			(1 f : forall x. p x -@ q (m x)) -> 
			(LExists p -@ LExists q)
		map f (LEvidence x y) = LEvidence (m x) (f y)
	\end{minted}
	\caption{Definition of \ident map for \ident LExists }
	\label{lst:def_lex_map}
\end{listing}
 
\subsection{Existential Crisis (Solution)}

This principle issue with using $M$ and $\Omega$ in practice is that

%%% Local Variables:
%%% mode: LaTeX
%%% TeX-master: "../main"
%%% End:
