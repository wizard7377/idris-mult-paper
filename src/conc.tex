\section{Conclusion}

\subsection*{Related Work}

\paragraph{GrTT and QTT}
While QTT, in particular as described here, is quite useful, it of course has its limits.
In particular, the work of languages like \granule to create a generalized notion of this in a way that can easily be infered and checked is important.
However, in terms of QTT (or even systems outside of it) this is far from complete.

\paragraph{The Syntax}
For instance, even with the \carat types, the syntax for this, and more generally linearity in general, tends to be a bit hard to use.
A question of how to integrate into the source syntax would be quite interesting.
Also, in general, one of the advantages of making an algebra part of the core language itself (as opposed to a construction on top of it) is that it makes it easier to create an inference engine for that language..

\paragraph{Bump Allocation}
QTT has been discussed as a potential theoretical model for ownership systems.
One of the more useful constructs in such a system is bump allocation.
With particular use seen in compilers, bump arena allocation, where memory is pre-allocated per phase, helps both seperate and simplify memory usage.
It is possible that a usage of $M$ types (given the fact that we know exactly how many times we need a value) as a form of modeling of arena allocation might be useful.


\subsection*{Acknowledgements}

Thank you to the Idris team for helping provide guidance and review for this.
In particular, I would like to thank Constantine for his help in the creation of $M$ types

\subsection*{Artifacts}

\label{sec:this_lib}
All Idris code mentioned here is either directly from or derived from the code in the Idris library \verb|idris-mult|, which may be found at \href{https://github.com/wizard7377/idris-mult}{its repository}\footnote{Some listings may be modified for readability}
