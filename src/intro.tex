\section{Introduction}

One of the more interesting developments in Programming Language Theory is Quantative Type Theory, or QTT.
Based off Girard's linear logic, it forms the basis of the core syntax of Idris 2's core language, and also as a starting point for that of Linear Haskell \cite{idris,linear_haskell,linear_logic,syntax_semantics_linear_types}.
It has many theoretical applications, including creating a more concrete interpretation of the concept of a "real world", constraining memory usage, and allowing for safer foriegn interfaces \needcite. 
Apart from just the theoretical intrest, QTT has the potential to serve as an underlying logic for languages like Rust, where reasoning about resources takes the forfront.

However, new developments in Graded Modal Type Theory (GrTT), in particular with Granule, serve to create a finer grained\footnote{Hence the name} notion of usage thatn QTT.
In GrTT, any natural number may serve as a usage, and in some cases even things that are first glance not natural numbers.
GrTT is part of a larger trend of "types with algbras" being used to create inferable, simple, and intuitive systems for models of various things.
Of these, some of the more notable ones are Koka, Effekt, and Flix, all of which seek to model effects with algebras \cite{flix_paper,effekt,koka_1}.

\subsection*{Preliminaries}

We write the morphisms in a given category $\to_{\mathcal{C}}$, where $\mathcal{C}$ is the category in question.
We define $1$ to be the category of Idris terms where the morphisms are all linear mappings, $0$ to be that where they are all erased, and $\omega$ for those that are unrestricted.
We write these arrows at $\to_1$, $\to_0$, and $\to_\omega$, respectively.

We also write isomorphisms in a certain category $\mathcal{C}$ as $\simeq_{\mathcal{C}}$, and in particular we have $\simeq_1$ as the isomorphism in the linear category.
We write implicit $\Pi$ types prefixed with a $\forall$, so, for instance, we have $\fa1 x : t . u$ as the equivalent of the Idris \verb|{1 x : t} -> u|.
We write $\times^{\mathcal{C}}$ for the product construction in $\mathcal{C}$, and for a given evidence of that construction we write that with the constructor $*^{\mathcal{C}}$
