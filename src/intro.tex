\section{Introduction}

Idris shows the practical applications of Quantitative type Theory, a branch of type theory based around the notion of \emph{usage}\cite{eb_idris_qtt_prac}.
In particular, the ability for multiplicities to model both modality in logic and resource usage in programs adds yet another way to reason about programs.

Idris 2's use of QTT as a core language has been one of its strong selling points. 
There are quite a few things that can be easily done with restricted bindings that cannot be done any other way.
For instance, they provide a good model for type variables, proof irrelevance at runtime, a less magical system of IO, among other properties.

Recently, however, there has been interest in Graded Modal Type Theory (GrTT), which allows for more dynamic restrictions on the usage of types, particularly allowing for usages greater than 1 \cite{quant_graded_modal}.
Rather than arguing for the use of GrTT in Idris (which would add complexity), we instead propose a construction, $M$, that can model graded quantities using only linear types.

Furthermore, a system is described that \emph{only} relies on types with multiplicities $0$ and $1$ to build a system (almost) as powerful as such as system with $\omega$.
We do this by using the exponential construction, here called $M$, to model arbitrary multiplicities\cite{dep_mult_dep_lin, inverse_of_type}
