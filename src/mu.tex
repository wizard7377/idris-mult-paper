\section{$M$ types}

To define $M$ types, we must first define \gls{type:source} types \cite{dep_mult_dep_lin}.
This is equivalent to the \verb|linear|  "copies" type, which it is modeled after \cite{inverse_of_type, idris_linear}. 
However, rather than \verb|Nil| and \verb|::|, we call its constructors instead $\mz$ (\verb|Exhaust|) and $\nsum$ (\verb|Provide|).

\begin{definition}
	\gls{type:source} is a polymorphic inductive type indexed by \gls{type:nats} and $t$, where is some $t : \gls{type:tpe}$.
	
	It has two constructors, $\mz : \gls{type:source} \ 0 \ x$, and $\nsum : (x : t) \proc \gls{type:source} \ n \ x \proc \gls{type:source} \ (S n) \ x$.
\end{definition}

Note that \gls{type:source} does not take a \gls{type:tpe} as an\footnote{explicit} arguement, but rather an \emph{element} of that type, $x$. 
Also note that by the rules of \gls{type:nats}, for a given $n$, there is \emph{only} one valid constructor \footnote{Specifically Peano's 8th Axiom} \needcite.

This will become incredibly useful, as we prove other facts about \gls{type:source}.

\begin{lemma}[Uniqueness of \gls{type:source}]
	\label{lem:unique}
	If $\Delta_0$ and $\Delta_1$ both have type $\gls{type:source} \ n \ x$, and that type is constructible, then $\Delta_0 \gls{sym:peq} \Delta_1$
\end{lemma}

\begin{proof}
	Let us induce on $n$.
	
	The base case $n \equiv 0$, thereby $\Delta_0$ and $\Delta_1$ are of type $\gls{type:source} \ 0 \ x$, which has only one constructor, $\mz$, which is equal to itself.
	
	The inductive case, then involves a proof that $a \nsum \Delta_0' \gls{sym:peq} b \nsum \Delta_1'$, where $\Delta_0' : \gls{type:source} \ n \ x$, $\Delta_1' : \gls{type:source} \ n \ x$.
	
	We note the type of $\nsum : (x : t) \proc \gls{type:source} \ n \ x \proc \gls{type:source} \ (S n) \ x$, and attempt to unify $\Delta_0' : \gls{type:source} \ n \ x$ with $\gls{type:source} \ n \ x$, and then infer the first argument must be of the form $x$.
	Thereby, $a \gls{sym:peq} x$.
	
	Do similar on $b$ and $\Delta_1'$, and we also get $b \gls{sym:peq} x$. 
	We then rewrite, using both these two results $x \nsum \Delta_0' \gls{sym:peq} x \nsum \Delta_1'$, then, we have $\Delta_0' \gls{sym:peq} \Delta_1'$.
	This is the induction hypothesis.
\end{proof}

\subsection{Witnessing Supplies}

While this would be quite useful for determining general copies of specific values, this isn't what we desire.
Rather, we want to model the notion of a graded modality in Idris.
For this, we need a way to ask for $n$-copies of a given type.

For this, we define another type $M$, which captures the notion of a generalized modality on types.

\begin{definition}
	$\tmu$ is a type indexed on a $\nat$ and $\gls{type:tpe}$. 
	It has one constructor, $S : (w :_0 t) \to \gls{type:source} \ n \ w \proc \tmu \ n \ t$.
	We refer to the erased $t$, $w$, as the witness.
\end{definition}

$M$ serves to "wrap around" a given \gls{type:source} by abstracting over the specific value, $w$.en
